\documentclass{article}
\usepackage[utf8]{inputenc}
\usepackage[spanish]{babel}
\usepackage{listings}
\usepackage{graphicx}
\graphicspath{ {images/} }
\usepackage{cite}

\begin{document}

\begin{titlepage}
    \begin{center}
        \vspace*{1cm}
            
        \Huge
        \textbf{Primeros Pasos}
            
        \vspace{0.5cm}
        \LARGE
        
            
        \vspace{1.5cm}
            
        \textbf{Julián David Quintero Marín}
		\textbf{Manuel Alejandro López Loaiza}
        \newline
        \newline
        \newline
        \newline

        
        
        \textbf{Informática II}
        
        \vfill
            
        \vspace{0.8cm}
  
        \Large
        Despartamento de Ingeniería Electrónica y Telecomunicaciones\\
        Universidad de Antioquia\\
        Medellín\\
        septiembre de 2021
            
    \end{center}
\end{titlepage}

\tableofcontents
\newpage
\section{Sección introductoria}\label{intro}
En esta actividad se expondrán ideas para llevar a cabo la elaboración del proyecto final del curso, que es un video juego. Estas ideas serán para dar claridad de lo que se quiere hacer para el proyecto final, es decir que las ideas están en función de ser modificadas más adelante.

\section{Sección de contenido} \label{contenido}
Mi idea para el videojuego es crear un mundo bajo el mar, en el que el jugador sea un tiburón y deba comer peces, medusas, pulpos, etc. (cualquier tipo de animal marino que esté en su cadena alimenticia). El jugador debe hacer que el tiburón permanezca comiendo, pues esto lo mantendrá con vida y por ende seguir en el juego, es decir que la vida del tiburón va bajando a medida que pasa el tiempo. 
El tiburón también puede perder vida de una forma más rápida si es que este llega a una profundidad determinada. (Aún no sé como haría esto, pero podría poner señales en el mapa que indiquen un límite para que el jugador pueda percatarse de este y sepa que debe acercarse más a la superficie).
Pensé además que puede haber un tiempo de 5min para que el jugador sobreviva, y si lo logra y se mantiene con vida, subirá al siguiente nivel. De lo contrario, tendrá que repetirlo hasta lograr llegar a los 5 min con vida.



\bibliographystyle{IEEEtran}

\end{document}\documentclass{article}
\usepackage[utf8]{inputenc}
\usepackage[spanish]{babel}
\usepackage{listings}
\usepackage{graphicx}
\graphicspath{ {images/} }
\usepackage{cite}

